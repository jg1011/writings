\documentclass{article} 

\usepackage[most]{tcolorbox}
\usepackage{xcolor}
\usepackage{enumerate}
\usepackage[a4paper, margin=1in]{geometry} 
\usepackage{titling}
\usepackage{amssymb}
\usepackage{lipsum}
\usepackage{mathtools}
\usepackage{amsthm} % For proof environments only, labelling of thms done with custom counter
\usepackage{amsmath}
\usepackage{amssymb}
\usepackage{textcomp} % TM logo, shits n giggles

%%% Misc commands %%%
\newcommand\iidsim{\stackrel{\mathclap{\tiny\mbox{i.i.d}}}{\sim}}
\newcommand\indist{\stackrel{\mathclap{\tiny\mbox{$\mathcal{D}$}}}{\longrightarrow}}
\newcommand\nidist{\stackrel{\mathclap{\tiny\mbox{$\mathcal{D}$}}}{\longleftarrow}}
\newcommand{\indep}{\perp\!\!\!\!\perp} 

\setlength{\parindent}{0pt} % Remove indentation upon new paragraph. 

%%% Title page information %%%
\title{The Probabilistic Method in Combinatorics}
\author{Jacob Green}
\date{\today}
\newcommand{\subtitle}{}
\newcommand{\institution}{Department of Mathematical Sciences, The University of Bath}
\newcommand{\keywords}{Combinatorics, Probability, Lovas\'{z} Local Lemma, Alteration}

\newcounter{statementcount}
\counterwithin{statementcount}{subsection}
\renewcommand{\thestatementcount}{\arabic{statementcount}}
\newcounter{claimcount} % Used for proof with lots of claims

\newtcolorbox{definition}[2][auto counter, number within=section]{
  colback=black!5!white, 
  colframe=black!50!white, 
  fonttitle=\bfseries, 
  coltitle=black,
  title=Definition \thestatementcount $\:$ (#2), 
  before upper = \refstepcounter{statementcount},
  #1, % Optional params
}

\newtcolorbox{lemma}[2][auto counter, number within=section]{
  colback=blue!5!white, 
  colframe=blue!50!white, 
  fonttitle=\bfseries, 
  coltitle=black, 
  title=Lemma \thestatementcount $\:$ (#2),
  before upper = \refstepcounter{statementcount} 
  #1, % Optional params
}

\newtcolorbox{example}[2][auto counter, number within=section]{
  colback=green!5!white, 
  colframe=green!50!white, 
  fonttitle=\bfseries, 
  coltitle=black, 
  title=Example \thestatementcount $\:$ (#2), 
  before upper = \refstepcounter{statementcount}
  #1, % Optional params
}

\newtcolorbox{problem}[2][auto counter, number within=section]{
  colback=purple!5!white, 
  colframe=purple!50!white, 
  fonttitle=\bfseries, 
  coltitle=black, 
  title=Problem \thestatementcount $\:$ (#2), 
  before upper = \refstepcounter{statementcount}
  #1, % Optional params
}

\newtcolorbox{theorem}[2][auto counter, number within=section]{
  colback=red!5!white,
  colframe=red!50!white, 
  fonttitle=\bfseries, 
  coltitle=black, 
  title=Theorem \thestatementcount $\:$ (#2), 
  before upper = \refstepcounter{statementcount},
  #1, % Optional params
}

\newtcolorbox{proposition}[2][auto counter, number within=section]{
  colback=purple!5!white,
  colframe=purple!50!white, 
  fonttitle=\bfseries, 
  coltitle=black, 
  title=Proposition \thestatementcount $\:$ (#2), 
  before upper = \refstepcounter{statementcount},
  #1, % Optional params
}

\newtcolorbox{corollary}[2][auto counter, number within=section]{
  colback=yellow!5!white,
  colframe=yellow!50!white, 
  fonttitle=\bfseries, 
  coltitle=black, 
  title=Corollary \thestatementcount $\:$ (#2),
  before upper = \refstepcounter{statementcount}, 
  #1, % Optional params
}

\newtcolorbox{remark}[2][auto counter, number within=section]{
  colback=black!5!white,
  colframe=black!50!white, 
  fonttitle=\bfseries, 
  coltitle=black, 
  title=Remark \thestatementcount $\:$ (#2),
  before upper = \refstepcounter{statementcount}, 
  #1, % Optional params
}

\newtcolorbox{exercise}[2][auto counter, number within=section]{
  colback=black!5!white,
  colframe=black!50!white, 
  fonttitle=\bfseries, 
  coltitle=black, 
  title=Exercise \thestatementcount $\:$ #2,
  before upper = \refstepcounter{statementcount}, 
  #1, % Optional params
}

\newtcolorbox[use counter=claimcount]{claim}[1][]{%
    colback=green!10, 
    colframe=green!50!black, 
    coltitle=black, 
    fonttitle=\bfseries, 
    boxrule=0.5mm, 
    colbacktitle=green!30,
    enhanced, % Allows additional options
    boxed title style={sharp corners}, 
    attach title to upper={},
    titlerule=0mm, 
    title=Claim: $\;$,
    before=\par\smallskip\noindent, 
    #1
}

\begin{document}

\begin{titlepage}
    \centering
    % Adding an image (optional)
    
    % Title
    {\Huge \bfseries \thetitle \par}
    \vspace{0.5cm}
    
    % Subtitle (if any)
    {\Large \subtitle \par}
    \vspace{1cm}
    
    % Author and institution
    {\large \theauthor \par}
    {\institution \par}
    \vspace{1cm}
    
    % Date
    {\large \thedate \par}
    \vspace{1.5cm}
    
    % Abstract section
    \begin{abstract}
        \lipsum[10]
    \end{abstract}
    \vspace{1cm}
    
    % Keywords section
    \textbf{Keywords:} \keywords
    \vfill % Pushes the following content to the bottom
    
    % Footer or further information
    \textit{}
\end{titlepage}

\newpage 

\setcounter{page}{1} % Start page numbering from 1 after title page

\section*{Preface}

\subsection*{Notation}

\begin{itemize}
    \item $\mathbb{N} = \{1, 2, \dots\}$ and $\mathbb{N}_0 = \mathbb{N} \cup \{0\}$
    \item $[n] = \{1, \dots, n\}$
    \item if $a$ is defined to be $b$ we may say $a := b$ (but I will often forget)
    \item $A^k = \overbrace{A \times \cdots \times A}^{k \text{ times}}$ for a set $A$, $\times$ being the Cartesian product. 
\end{itemize}

\newpage

\tableofcontents

\newpage

\section{Introduction}

Typically, the probabilistic method is used to prove the existence of a construction with desirable properties 
without actually constructing it. We do this by showing it exists with some positive probability.   \\

A simple tool for doing so is the following. 

\begin{lemma}[]{a random variable must take its mean value}
    Let $X$ be an integer-valued random variable. Then \[\mathbb{P}(X \geq \mathbb{E}[X]) > 0\]
\end{lemma}

\begin{proof}
    Suppose $\mathbb{P}(X \geq \mathbb{E}[X]) = 0$. Then, 
    \begin{align*}
        \mathbb{E}[X] &= \sum_{x = - \infty}^{\mathbb{E}[X] - 1}x\mathbb{P}(X = x) + \sum_{x = \mathbb{E}[X]}^\infty 
        x\mathbb{P}(X = x) \\ &\leq (\mathbb{E}[X] - 1)\sum_{x = - \infty}^{\mathbb{E}[X] - 1}\mathbb{P}(X = x) + 
        \sum_{x = \mathbb{E}[X]}^\infty x\mathbb{P}(X \geq x) \\ &= (\mathbb{E}[X] - 1)\mathbb{P}(X \leq \mathbb{E}[X] - 1)
        = \mathbb{E}[X] - 1
    \end{align*}
    Absurd!
\end{proof}

Clearly our argument works for any $X$ taking values on a countable set. A continuous analogue also exists. \\ 

We can use this to prove any graph has a large bipartite subgraph. The idea will be to $2$-colour its vertices 
uniformly, using the fact a bipartite subgraph is a graph where we can two colour the vertices such that 
every edge has endpoints with distinct colours. 

\begin{proposition}[]{large bipartite subgraph}
    Every graph with $m$ edges has a bipartite subgraph with at least $m/2$ edges.
\end{proposition}

\begin{proof}
    Let our graph be $G = G(V, E)$ and assign each vertex a colour from $\{1, 2\}$. Let $E^\prime \subseteq E$ be 
    the set of edges with distinctly coloured endpoints. Then, enumerating $E = \{e_1, \dots, e_m\}$ and letting 
    $A_i$ be the event edge $e_i$ has distinctly coloured endpoints, 
    \[\mathbb{E}[|E^\prime|] = \mathbb{E}\left[\sum_{i = 1}^m{\bf 1}(A_i)\right] = \sum_{i=1}^m \mathbb{P}(A_i) = \frac{1}{2}m\]
    via the linearity of expectations and the fact $\mathbb{E}[{\bf 1}(A)] = \mathbb{P}(A)$ for any event $A$, 
    where ${\bf 1}(\cdot)$ is the indicator function\footnote{That is, ${\bf 1}(A) = 1$ where $A$ is true and $0$ elsewhere}. 
    Hence, by our previous lemma 
    \[\mathbb{P}(|E^\prime| \geq m/2) > 0\] 
    and we have our large bipartite subgraph.
\end{proof}

Linearity of expectations as a tool is extremely prevelant in probabilistic combinatorics, so much so 
we dedicate an entire section to its enumerable applications. \\ 

\subsubsection{Lower Ramsey Bounds}

\begin{definition}[]{Ramsey number}
    Let $k, \ell \in \mathbb{N}$. Then the Ramsey number $R(k, \ell)$ is the smallest $n \in \mathbb{N}$ such that 
    in every $2$-colouring of $K_n$ we have a monochromatic $K_k$ of colour or a monochromatic $K_\ell$ of colour 2.
\end{definition}

For example, one can prove $R(3,3) = 6$ with the Pidgeonhole principle. WLOG we can choose a vertex such that 
$3$ of the edges from this vertex have colour $1$. Now consider the triangle formed by the edges connecting these $3$
endpoints. If any one of the edges in this triangle have colour $1$ then we have a monochromatic triangle, if not 
then we also have a monochromatic triangle, giving $R(3,3) \leq 6$. Then colour $K_5$ nicely to prove $R(3,3) > 5$. 

\begin{remark}[]{a nice interpretation}
    Proving $R(3,3) \geq 6$ is a classical olympiad problem, often stated as ``At a party with 
    $6$ people, there are $3$ people that all know eachother or $3$ people that do not know one another.'' 
\end{remark}

Ramsey first proved that these numbers exist. 

\begin{theorem}[]{Ramsey's Theorem - 1929}
    For all $k, \ell \in \mathbb{N}$, $R(k, \ell)$ exists and is finite. 
\end{theorem}

\begin{proof}
    Omitted. 
\end{proof}

Estimating the Ramsey number is a fundamental problem in Ramsey theory, a vast subfield of graph theory. We 
will prove 3 quantitative lower bounds for the diagonal Ramsey number $R(k, k)$, each serving as an exhibition 
of an important technique in probabilistic combinatorics. \\ 

The first technique we use will be the so called union bound, often referred to by probability theorists as 
Boole's inequality. 

\begin{lemma}[]{union bound / Boole's inequality}
    Let $A_1, \dots, A_n$ be events on a probability space $(\omega, \mathcal{F}, \mathbb{P})$. Then, 
    \[\mathbb{P}\left(\bigcup_{i = 1}^n A_i\right) \leq \sum_{i = 1}^n \mathbb{P}(A_i)\]
\end{lemma}

We'll prove the $n=2$ case, which gives the $n \in \mathbb{N}$ case immediately by induction. 

\begin{proof}
    Write $A_1 \cup A_2 = (A_1 \setminus A_2) \cup (A_2 \setminus A_1)$ as a disjoint union, noting $A_i \setminus A_j 
    \subseteq A_i$ for $i \neq j$. Then,
    \[\mathbb{P}(A_1 \cup A_2) = \mathbb{P}(A_1 \setminus A_2) + \mathbb{P}(A_2 \setminus A_1) 
    \leq \mathbb{P}(A_1) + \mathbb{P}(A_2)\] as required.
\end{proof}

Alternatively, the $2$ variable inclusion-exclusion principle gives this immediately. \\ 

We may use this to bound unions from above, giving us a way to bound the probability of some ``bad'' event 
$A_i$ from a set of ``bad'' events $A_1, \dots, A_n$ occuring. Erd\H{o}s used this methodology to obtain the 
following diagonal Ramsey lower bound, one of the first known applications of the probabilistic method in combinatorics. 

\begin{proposition}[]{Ramsey LB via union bound - Erd\H{o}s 1947}
    If $\binom{n}{k}2^{1 - \binom{k}{2}} < 1$ then $R(k, k) > n$. 
\end{proposition}

We will consider random $2$-colourings of the edges of $K_n$, showing with positive probability if 
$k$ satisfies $\binom{n}{k}2^{1 - \binom{k}{2}} < 1$ then $K_n$ has no monochromatic $K_k$. 

\begin{proof}
    Independently and uniformly $2$-colour the edges of $K_n$ at random. There are $\binom{n}{k}$ 
    subgraphs isomorphic to $K_k$, and these are exactly the graphs obtained by taking the subgraph induced by 
    keeping $k$ edges from $K_n$. Let $m = \binom{n}{k}$ and $A_1, \dots, A_{m}$ be the events these graphs are monochromatic. 
    Then, 
    \[\mathbb{P}\left(\bigcup_{i = 1}^{m}A_i\right) \leq \sum_{i = 1}^{m}\mathbb{P}(A_i) = 
    \binom{n}{k} \times (2 \times 2^{- \binom{n}{k}})\]
    so that, via DeMorgan's laws,
    \[\binom{n}{k}2^{1 - \binom{k}{2}} < 1 \Longrightarrow \mathbb{P}\left(\bigcap_{i = 1}^{m}A_i^c\right) > 0\]
    giving us a $2$-colouring of the edges of $K_n$ with no monochromatic $K_n$. 
\end{proof}

This bound doesn't, at first, tell us anything about the size of $k$ relative to $n$. To deduce this fact we'll 
need to do some work with asymptotics.

\begin{proposition}[]{quantitative Erd\H{o}s lower bound}
    For $k \in \mathbb{N}$, 
    \[R(k, k) > \left(\frac{1}{\sqrt{e}} + o(1)\right)k2^{k/2}\]
\end{proposition}

We will work with Stirling's approximation.

\begin{proof}
    I can't get it to bloody work :/
\end{proof}

The next approach we'll take is via the so called method of {\it alteration}. In this approach, generally,
we start with a random construction, fix the undesirable features of said construction and prove our construction 
still has an interesting size. This approach is so important we'll dedicate a whole section to it. \\ 

In this application, as before, we'll take a $2$-colouring of the edges of $K_n$ independently and uniformly. This 
time though, instead of just analysing these colourings, we'll turn them into graphs $K_X$ that certainly contain no 
monochromatic $K_k$ and analyse the size of $X$. 

\begin{proposition}[]{Ramsey LB via alteration - unknown}
    For all $k, n \in \mathbb{N}$, $R(k,k) > n - \binom{n}{k}2^{1-\binom{k}{2}}$
\end{proposition}

\begin{proof}
    $2$-colour the edges of the clique $K_n$ independently and uniformly at random. Iteratively delete a vertex 
    from each monochromatic $K_k$. Let $m = \binom{n}{k}$ and let $A_1, \dots, A_m$ be the events the $i^\text{th}$ 
    $k$-clique is monochromatic after the initial colouring (before the deletion of any vertices). Let $X$ be the 
    number of vertices we delete during the deletion process. Then, 
    \[\mathbb{E}[X] = \mathbb{E}\left[\sum_{i = 1}^m {\bf 1}(A_i)\right] = m\mathbb{P}(A_i) = \binom{n}{k}2^{1-\binom{k}{2}}\]
    so, as we have $n - X$ vertices remaining after the alteration, we have 
    \[\mathbb{P}\left(n - X \geq n - \binom{n}{k}2^{1-\binom{k}{2}}\right) > 0\]
    i.e. a $2$-colouring of the edges of the clique size $n - \binom{n}{k}2^{1-\binom{k}{2}}$ containing no monochromatic 
    $K_k$. 
\end{proof}

Notice, again we used linearity of expectations here, as well as lemma 2.1. This will often be the case in 
alteration proofs. \\ 

As before, we can optimise $n$ for fixed $k$ to gain a quantitative bound.

\begin{proposition}[]{quantitative Ramsey LB via alteration}
    \[R(k, k) > \left(\frac{1}{e} + o(1)\right)k2^{k/2}\]
\end{proposition}

\begin{proof}

\end{proof}

Note this improves the Erd\H{o}s bound by a factor of $\frac{1}{\sqrt{e}}$. \\

Now we introduce our third method, again, warranting its own section. The idea is that the so called ``bad'' events
$A_1, \dots, A_n$ we want to avoid are often only weakly dependent, and we have a great bound for the probability 
that none of the ``bad'' events occur in such a setup (courtesy of Lovas\'{z}).

\begin{lemma}[]{Lovas\'{z} Local Lemma - random variable model}
    Let $X_1, \dots, X_n$ be independent random variables. Let $B_1, \dots, B_m \subset [n]$ and, for each
    $1 \leq i \leq m$, let $A_i$ be an event depending only on the random variables $\{X_j, j \in B_i\}$. \\ 
    
    If for each $i$, $B_i \cap B_j \neq \emptyset$ for at most $d$ other $B_j$'s and $\mathbb{P}(A_i) \leq 1/e(d+1)$ 
    then with positive probability none of the $A_i$ occur. 
\end{lemma}

We will save the proof of this for our section dedicated to this lemma. \\

Now lets use this to obtain another improvement on our Ramsey lower bound. As before we'll take random $2$-colourings 
and consider the events $A_1, \dots, A_m$ where clique $i$ is monochromatic, but, observing the weak dependence 
between these events (and letting the colour of the edges be our random variables) this makes a perfect candidate 
for the Lovas\'{z} Local Lemma. 

\begin{proposition}[]{Ramsey LB via LLL - Spencer 1977}
    If $\left(\binom{k}{2}\binom{n}{k-2} + 1\right)2^{1 - \binom{k}{2}} < \frac{1}{e}$ then $R(k, k) > n$. 
\end{proposition}

\begin{proof}
    $2$-Colour the edges of $K_n$ independently and uniformly at random, letting the colour of $1 \leq i \leq 
    \binom{n}{2}=: N$ be the random variable $X_i$. Enumerate the $k$-cliques of $K_n$ as $G_1, \dots, G_M$ where 
    $M := \binom{n}{k}$ and let $A_j: 1 \leq j \leq M$ be the event $G_i$ is monochromatic. Let 
    $B_j \subset [N]: 1 \leq j \leq M$ be the edges of $G_i$. Clearly $A_i$ depends only on
    the random variables $\{X_j : j \in B_i\}$. \\

    Now $B_i \cap B_j \neq \emptyset$ exactly when $G_i$ and $G_j$ share at least one edge. This is equivalent 
    (edges have two endpoints and two vertices make an edge!) to $G_i$ and $G_j$ sharing at least two vertices. 
    Fixing $G_i$, we may count the number of $G_j$ satisfying this condition by counting the number of ways to 
    choose $2$ vertices from $G_i$ (our shared vertices) and multiplying by the number of ways to choose the remaining 
    $k-2$ vertices from the vertices of $K_n$ (noting we may choose from all $n$, as $B_i$ can share any number of 
    vertices with $B_j$) giving us a candidate $d$ of 
    \[d = \binom{k}{2}\binom{n}{k-2}\]
    Now, as we've already seen, $\mathbb{P}(A_i) = 2^{1 - \binom{k}{2}}$, so to apply the Lovas\'{z} local lemma 
    we need 
    \[\binom{k}{2}\binom{n}{k-2}2^{1 - \binom{k}{2}} < \frac{1}{e}\]
    If this inequality in $n$ and $k$ holds, we have with positive probability a colouring of the edges of $K_n$
    with no monochromatic $k$-clique, and thus $R(k, k) > n$. 
\end{proof}

We may optimise one final time for $n$ in $k$ to get a quantitative bound.

\begin{proposition}[]{quantitative Ramsey LB via LLL}
    \[R(k,k) > \left(\frac{\sqrt{2}}{e} + o(1)\right)k2^{k/2}\]
\end{proposition}

\begin{proof}
    
\end{proof}

This is improves our alteration bound by a factor of $\sqrt{2}$, and is currently the best known lower bound for 
the diagonal Ramsey number. Philosophically, it is interesting to note that all three of these different techniques 
only improve our lower bound by a constant factor. With these methods, we always obtain a $O(k2^{k/2})$ lower bound. \\

We are very far from knowing the exact diagonal ramsey number. Recently$\dots$ (Campos upper bound exponential 
dialogue via a kind of alteration). 

\subsubsection{Set Systems}

A {\it set system} is simply a family $\mathcal{F}$ of subsets of some space $X$. Often we take $X = [n]$. In 
{\it extremal set theory} we are usually concerned with finding the maximal/minimal size of a set system that 
has some desirable property. \\ 

The first type of set system we study are the so called {\it antichains}. These are families that are pairwise 
incomparable by containment, that is $\forall A \in \mathcal{F} \; \nexists B \in \mathcal{F}$ with $ A \subseteq B$. \\

We will write $\binom{[n]}{k}$ for the family of $k$-element subsets of $[n]$. 

\begin{claim}
    The family $\binom{[n]}{k}$ is an antichain for any $0 \leq k \leq n$
\end{claim}

\begin{proof}
    Suppose there are $k$ element subsets of $A, B \subseteq [n]$ with $A \subseteq B$. Then, as $|A| = k = |B|$ 
    we have $A = B$ forced but, as our family is a set, its elements must be unique. Contradiction!
\end{proof}

The size of $\binom{[n]}{k}$ is $\binom{n}{k}$, which is maximised by taking $k = \lfloor n/2 \rfloor$. It turns 
out the family $\binom{[n]}{\lfloor n/2 \rfloor}$ is our biggest antichain in $[n]$. 

\begin{theorem}[]{Sperner's theorem - 1928}
    If $\mathcal{F}$ is an antichain on $[n]$, then $|\mathcal{F}| \leq \binom{n}{\lfloor n/2 \rfloor}$
\end{theorem}

Rather than proving this we will prove a stronger result independently discovered by several mathematicians, 
often referred to as the LYM inequality. 

\begin{theorem}[]{LYM inequality - 1954/1963/1965/1966}
    If $\mathcal{F}$ is an antichain on $[n]$ then 
    \[\sum_{A \in \mathcal{F}}\binom{n}{|A|}^{-1} \leq 1\]
\end{theorem}

Note this immediately uncovers Sperner's theorem as $\binom{n}{|A|} \leq \binom{n}{\lfloor n/2\rfloor}$. \\

The idea will be to consider random permutations of $n$ and the canonical chain generated from this permutation. 
If we have two events in said canonical chain we no longer have an anti-chain, so an easy disjoint probabilities 
sum arises. 

\begin{proof}
    Choose a permutation $\sigma$ or $[n]$ uniformly at random and consider the canonical chain 
    \[\mathcal{S}(\sigma) = \{\emptyset, \{\sigma(1)\}, \{\sigma(1), \sigma(2)\}, \dots, \{\sigma(1), \dots, \sigma(n)\}\}\]
    For subsets $A \subset [n]$ let $E_A$ be the event $A \in \mathcal{S}(\sigma)$. Then we have $A \in \mathcal{S}(\sigma)$ 
    if and only if the elements of $A$ are exactly the first $|A|$ elements in the listed permutation 
    $(\sigma(1), \dots, \sigma(|A|))$ up to reordering, leaving the remaining $n - |A|$ elements of $[n]$ to be the 
    remaining elements $(\sigma(|A| + 1), \dots, \sigma(n))$ up to reordering. Thus, as we have $|A|!(n-|A|!)$ 
    such permutations, we have \[\mathbb{P}(E_A) = \frac{|A|!(n-|A|)!}{n!} = \binom{n}{|A|}^{-1}\]
    Now suppose $\mathcal{F}$ is an antichain. Then clearly we have at most one $A \in \mathcal{F}$ with 
    $A \in \mathcal{S}(\sigma)$, hence $\{E_A : A \in \mathcal{F}\}$ are disjoint events and we have 
    \[1 \geq \mathbb{P}\left(\bigcup_{A \in \mathcal{F}}E_A\right) = \sum_{A \in \mathcal{F}}\mathbb{P}(E_A)
    = \sum_{A \in \mathcal{F}}\binom{n}{|A|}^{-1}\]
\end{proof}

Here our probabilistic tool was actually just an axiom of measure, namely the sum of the measure of disjoint 
subsets is equal the measure of the union of said subsets. One could quite easily rephrase this as a counting argument 
by using disjointness and the fact we have at most $n!$ permutations, then dividing through by $n!$ at the very end. 

\subsubsection*{Exercises}

\newpage

\subsection{Linearity of Expectation}

\subsubsection*{Exercises}

\newpage

\subsection{Alteration}

\subsubsection*{Exercises}

\newpage

\subsection{Lovas\'{z} Local Lemma}

\subsubsection*{Exercises}

\newpage

\section{Solutions}

\end{document}