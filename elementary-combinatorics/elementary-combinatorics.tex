\documentclass{article} 

\usepackage[most]{tcolorbox}
\usepackage{xcolor}
\usepackage{enumerate}
\usepackage[a4paper, margin=1in]{geometry} 
\usepackage{titling}
\usepackage{amssymb}
\usepackage{lipsum}
\usepackage{mathtools}
\usepackage{amsthm} % For proof environments only, labelling of thms done with custom counter
\usepackage{amsmath}
\usepackage{amssymb}
\usepackage{textcomp} % TM logo, shits n giggles
\usepackage{hyperref}

%%% Misc commands %%%
\newcommand\iidsim{\stackrel{\mathclap{\tiny\mbox{i.i.d}}}{\sim}}
\newcommand\indist{\stackrel{\mathclap{\tiny\mbox{$\mathcal{D}$}}}{\longrightarrow}}
\newcommand\nidist{\stackrel{\mathclap{\tiny\mbox{$\mathcal{D}$}}}{\longleftarrow}}
\newcommand{\indep}{\perp\!\!\!\!\perp} 

\setlength{\parindent}{0pt} % Remove indentation upon new paragraph. 

%%% Title page information %%%
\title{Elementary Combinatorics}
\author{Jacob Green}
\date{\today}
\newcommand{\subtitle}{}
\newcommand{\institution}{Department of Mathematical Sciences, The University of Bath}
\newcommand{\keywords}{Combinatorics, Binomial Coefficients, Pigeonhole Principle, Inclusion Exclusion Principle, 
Generating Functions, Graph Theory}

\newcounter{statementcount}
\counterwithin{statementcount}{subsection}
\renewcommand{\thestatementcount}{\thesection.\arabic{statementcount}}
\newcounter{claimcount} % Used for proof with lots of claims

\newtcolorbox{definition}[2][auto counter, number within=section]{
  colback=black!5!white, 
  colframe=black!50!white, 
  fonttitle=\bfseries, 
  coltitle=black,
  title=Definition \thestatementcount $\:$ (#2), 
  before upper = \refstepcounter{statementcount},
  #1, % Optional params
}

\newtcolorbox{lemma}[2][auto counter, number within=section]{
  colback=blue!5!white, 
  colframe=blue!50!white, 
  fonttitle=\bfseries, 
  coltitle=black, 
  title=Lemma \thestatementcount $\:$ (#2),
  before upper = \refstepcounter{statementcount} 
  #1, % Optional params
}

\newtcolorbox{example}[2][auto counter, number within=section]{
  colback=green!5!white, 
  colframe=green!50!white, 
  fonttitle=\bfseries, 
  coltitle=black, 
  title=Example \thestatementcount $\:$ (#2), 
  before upper = \refstepcounter{statementcount}
  #1, % Optional params
}

\newtcolorbox{problem}[2][auto counter, number within=section]{
  colback=purple!5!white, 
  colframe=purple!50!white, 
  fonttitle=\bfseries, 
  coltitle=black, 
  title=Problem \thestatementcount $\:$ (#2), 
  before upper = \refstepcounter{statementcount}
  #1, % Optional params
}

\newtcolorbox{theorem}[2][auto counter, number within=section]{
  colback=red!5!white,
  colframe=red!50!white, 
  fonttitle=\bfseries, 
  coltitle=black, 
  title=Theorem \thestatementcount $\:$ (#2), 
  before upper = \refstepcounter{statementcount},
  #1, % Optional params
}

\newtcolorbox{proposition}[2][auto counter, number within=section]{
  colback=purple!5!white,
  colframe=purple!50!white, 
  fonttitle=\bfseries, 
  coltitle=black, 
  title=Proposition \thestatementcount $\:$ (#2), 
  before upper = \refstepcounter{statementcount},
  #1, % Optional params
}

\newtcolorbox{corollary}[2][auto counter, number within=section]{
  colback=yellow!5!white,
  colframe=yellow!50!white, 
  fonttitle=\bfseries, 
  coltitle=black, 
  title=Corollary \thestatementcount $\:$ (#2),
  before upper = \refstepcounter{statementcount}, 
  #1, % Optional params
}

\newtcolorbox{remark}[2][auto counter, number within=section]{
  colback=black!5!white,
  colframe=black!50!white, 
  fonttitle=\bfseries, 
  coltitle=black, 
  title=Remark \thestatementcount $\:$ (#2),
  before upper = \refstepcounter{statementcount}, 
  #1, % Optional params
}

\newtcolorbox{exercise}[2][auto counter, number within=section]{
  colback=black!5!white,
  colframe=black!50!white, 
  fonttitle=\bfseries, 
  coltitle=black, 
  title=Exercise \thestatementcount $\:$ #2,
  before upper = \refstepcounter{statementcount}, 
  #1, % Optional params
}

\newtcolorbox[use counter=claimcount]{claim}[1][]{%
    colback=green!10, 
    colframe=green!50!black, 
    coltitle=black, 
    fonttitle=\bfseries, 
    boxrule=0.5mm, 
    colbacktitle=green!30,
    enhanced, % Allows additional options
    boxed title style={sharp corners}, 
    attach title to upper={},
    titlerule=0mm, 
    title=Claim: $\;$,
    before=\par\smallskip\noindent, 
    #1
}

\begin{document}

\begin{titlepage}
    \centering
    % Adding an image (optional)
    
    % Title
    {\Huge \bfseries \thetitle \par}
    \vspace{0.5cm}
    
    % Subtitle (if any)
    {\Large \subtitle \par}
    \vspace{1cm}
    
    % Author and institution
    {\large \theauthor \par}
    {\institution \par}
    \vspace{1cm}
    
    % Date
    {\large \thedate \par}
    \vspace{1.5cm}
    
    % Abstract section
    \begin{abstract}
        \lipsum[10]
    \end{abstract}
    \vspace{1cm}
    
    % Keywords section
    \textbf{Keywords:} \keywords
    \vfill % Pushes the following content to the bottom
    
    % Footer or further information
    \textit{}
\end{titlepage}

\newpage 

\setcounter{page}{1} % Start page numbering from 1 after title page

\section*{Preface}

This handout is designed for undergraduates, particularly those with no olympiad training, to learn the elementary theory 
of combinatorics. Ideally, these notes will provide a good exposure to the basic tricks that will be ``trivial'' 
in graduate combinatorics courses, and will further motivate the reader to go on and study these graduate courses. 
Some of my other handouts will explore this material, particularly those on random graphs, random geometric graphs 
and the probabilistic method in combinatorics. \\

These notes could also be read by a motivated high school student training for olympiads, I also refer such readers 
to \cite{di2014problem} and \hyperlink{https://yufeizhao.com/olympiad/}{Yufei Zhao's handouts}. I'm sure there are 
many other great resources if you do a bit of googling, I never did any olympiad training though so I wouldn't know! 

\subsection*{Notation}

\begin{itemize}
    \item $\mathbb{N} = \{1, 2, \dots\}$ and $\mathbb{N}_0 = \mathbb{N} \cup \{0\}$
    \item $[n] = \{1, \dots, n\}$
    \item if $a$ is defined to be $b$ we may say $a := b$ (but I will often forget)
    \item $A^k = \overbrace{A \times \cdots \times A}^{k \text{ times}}$ for a set $A$, $\times$ being the Cartesian product. 
\end{itemize}

\newpage

\tableofcontents

\newpage

\section{Introduction}

\setcounter{statementcount}{1}

In combinatorics we are concerned with counting the number of objects from a collection. For example, we may count the positive integers divisible
by $3$ that are less than $1000$. Here the objects are positive integers are the objects and the collection is those objects that are both divisible 
by $3$ and less than $1000$. \\

As a set is just is collection of objects, we will often phrase things set theoretically. \\

To begin counting anything interesting, we'll need the following principles. The first two let us count things inductively, the final 
says it doesn't matter what order we count each element in. These are essentially common sense principles. Regardless, being mathematicians, 
we will prove these.

\begin{theorem}[]{addition principle}
    Let $n \in \mathbb{N}$. Let $A_1, \dots, A_n$ be finite sets. One has\footnote{$A_1, \dots, A_n$ are said to be disjoint if 
    $\cap_{i=1}^n A_i = \emptyset$}
    \[A_1, \dots, A_n \text{ disjoint} \quad \Longrightarrow \quad |A_1 \cup \cdots \cup A_n| = |A_1| + \cdots + |A_n|\]
\end{theorem}

To prove this, we simply count both sides one term at a time.

\begin{proof}
    Let $x \in \cup_{i=1}^n A_i$. Then $x \in A_i$ for exactly one $1 \leq i \leq n$ and hence 
    \[|A_1| + \cdots + |A_n| =\sum_{x \in A_1}1 + \cdots + \sum_{x \in A_n}1 = \sum_{x \in \cup_{i=1}^n A_i}1 = |A_1 \cup \cdots \cup A_n|\] 
    by counting both sides term by term.
\end{proof}

\begin{theorem}[]{multiplication principle}
    Let $n \in \mathbb{N}$. Let $A_1, \dots, A_n$ be finite sets. One has
    \[|A_1 \times \cdots \times A_n| = |A_1| \times \cdots \times |A_n|\]
\end{theorem}

This time we count terms in each slot recursively.

\begin{proof}
    Consider an arbitrary element $x = (x_1, \dots, x_n) \in A_1 \times \cdots \times A_n$ and let $1 \leq i \leq n$. Then fixing $x_j, j \neq i$
    we have $|A_i|$ possible choices for $x_i$. Hence, we have 
    \[|A_1 \times \cdots \times A_n| = \sum_{i=1}^n\sum_{x \in A_i}\left|\cap_{j \neq i}A_j\right| 
    = \sum_{i=1}^n |A_i|\sum_{x \in A_i}\left|\cap_{j \neq i}A_j\right| = \cdots = |A_1| \times \cdots \times |A_n|\]
    by repeated application of this reasoning.
\end{proof}

\begin{theorem}[]{bijection principle}
    Let $A$ and $B$ be sets. If there is a bijection (i.e. one-to-one mapping) $\pi: A \to B$ then $|A| = |B|$
\end{theorem}

\begin{proof}
    This is a truism. 
\end{proof}

The addition principle can be thought of as emptying two containers of $n$ and $m$ wine bottles and containing how many bottles you have 
in total ($n+m$), whereas the multiplication principle can be thought of as counting the number of ways of pairing $b$ boys with $g$ girls. 
The bijection principle says if we take $s$ students and select them one by one at random, $t$ times, we'll always have selected $t$ students. 
We will often look for such interpretations in combinatorics. \\

With just these principles, we can count all the following quantities. 

\begin{problem}[]{$k$-words with repetitions}
    How many words length $k$ can be formed from an alphabet with $n$ letters.
\end{problem}

\begin{proof}[Solution]
    To count the number of words with repetitions allowed, we sample with replacement to get
    \[\underbrace{n \times \cdots \times n}_{k \text{ times}} = n^k\]
    $k$-words (or $k$-sequences or $k$-tuples) on $n$ symbols by the multiplication principle. \\
\end{proof}

\begin{problem}[]{$k$-words without repetitions}
    How many words length $k$ with distinct letters can be formed from an alphabet with $n$ letters. 
\end{problem}

\begin{proof}[Solution]
    For $k$-words without repetition, we sample without replacement for each letter to get
    \[n \times (n-1) \times \cdots \times (n-k+1) =: n^{\underline{k}}\]
    $k$-words without repetition on $n$ symbols by the multiplication principle.\footnote{
        We call $n^{\underline{k}}$ the falling factorial of $n$}
\end{proof}

If we were to consider words of length $n$ on $n$ symbols without repetition, then we would be {\it permuting} (i.e. reordering/relabelling) 
the $n$ symbols. Hence we have $n \times (n-1) \times \cdots \times 1 =: n!$ {\it permutations} on $n$ symbols. 
Using this notation we have $n^{\underline{k}} = n!/(n-k)!$. \\ 

Suppose we wish to count the number of subsets length $2$ on $n$ symbols. We have $n(n-1)$ ways of choosing 
$2$ unique elements (i.e. $2$-words without repetition) and we have $2!$ ways of arranging the $2$ unique elements. 
Hence, as each each $2$ element subset we count gives rise to $2!=2$ $2$-words, we may count these 
and divide by $2$, obtaining $\frac{1}{2}n(n-1)$ unique $2$-element subsets\footnote{This gives a clever proof that 
the product of two consecutive integers is divisible by two. We can use combinatorics to prove many number theoretic 
facts, and this is far from the only example we'll see in this handout.}. This method is often referred to as overcounting and correcting, and 
gives us another way to count unknown objects with known objects.  

\begin{problem}[]{$k$-subsets}
    How many subsets size $k$ are there of a set size $n$?
\end{problem}

\begin{proof}[Solution]
    Let $x$ be the desired quantity. For each subset size $k$, we may reorder it's elements in $k!$ ways and we 
    may select $k$ unique elements by counting $k$-words on $n$ symbols without repetition, thus 
    \[x \cdot k! = \frac{n!}{(n-k)!}\] and dividing by $k!$ gives $x = n!/k!(n-k)! =: \binom{n}{k}$
\end{proof}

We call $\binom{n}{k}$ the {\it binomial coefficient}, and would say this as ``$n$ choose $k$'' (as this counts the ways of 
choosing $k$ things from $n$ things). We will dedicate an entire section (in fact, next section) to these. \\ 

In this next problem we will first encounter the idea of ``stars and bars'' counting, which gives us a way of counting 
the ways to group $n$ indistinguishable objects into $k$ groups. This has many novel uses, like the one we're 
about to encounter. 

\begin{problem}[]{$k$-multisubsets}
    How many multisubsets\footnote{a multiset is a set that allows repetition, see https://en.wikipedia.org/wiki/Multiset} of size $k$ are there
    in a set of $n$ objects?
\end{problem}

\begin{proof}[Solution]
    Let $X = \{x_1, \dots, x_n\}$. We call the number of times $x_i, 1 \leq i \leq n$ appears in a multisubset of $X$ the {\it multiplicity} of $x_i$.
    We know that the size of a multiset is the sum of its multiplicities, thus it suffices to count solutions $(\ell_1, \dots, \ell_n)$ of 
    nonnegative integers to 
    \[\ell_1 + \cdots + \ell_n = k\] 
    which by setting $\ell^\prime := \ell + 1$ is the same as counting positive integer solutions to 
    \[\ell^\prime_1 + \cdots + \ell^\prime_n = k + n\] Now write this as \[\underbrace{1 + \cdots + 1}_{\ell_1^\prime \text{ times}} + 
    \cdots + \underbrace{1 + \cdots + 1}_{\ell_n^\prime \text{ times}} = k + n\]
    We are tasked with dividing up the $k + n$ ones into $n$ groups, which can be done by insering $n-1$ stars in between two ones (not the edges
    as we require $n$ nonempty collections of ones) to indicate the beginning of a new group. This gives $\binom{n+k-1}{n-1} = \binom{n+k-1}{k}$
    possibilities.
\end{proof}

\begin{problem}[]{$k$-partitions with a given sizing}
    How many ways are there to divide a set partition a set $X$ of size $n$ into sets $X_1, \dots, X_k$ with size $n_1, \dots, n_k$ respectively 
    (where $n_1 + \cdots + n_k = n$)?
\end{problem}

\begin{proof}[Solution]
    There are 
    \[\binom{n - n_1 - \cdots - n_{i-1}}{n_i}\]
    ways of choosing the $X_i$ once $X_1, \dots, X_{i-1}$ have been chosen, and 
    \[\binom{n}{k}\binom{n - k}{m} = \frac{n!}{k!m!(n-k-m)!}\] 
    So all together one obtains 
    \[\binom{n}{n_1}\binom{n-n_1}{n_2}\cdots\binom{n - (n_1 + \cdot + n_{k-1})}{n_k} = \frac{n!}{n_1!\cdots n_k!} =: \binom{n}{n_1, \dots, n_k}\]
    such partitions.\footnote{The quantity $\binom{n}{n_1, \dots, n_k}$ is called the multinomial coefficient}
\end{proof}

\subsubsection*{Exercises}

\begin{exercise}[]{}
    In how many ways can $n$ people be seated at a circular table with $n$ seats.
\end{exercise}

\begin{exercise}[]{}
    How many subsets are there of the set $\{1, 2, \dots, n\} =: [n]$? How many are of even size? 
\end{exercise}

\begin{exercise}[]{}
    There are $2n$ people at a party. How many ways can the $2n$ people split into $n$ pairs? 
\end{exercise}

\begin{exercise}[]{}
    A robot is placed on the bottom left of an $n \times n$ chessboard. The robot has two moves, go vertically up one square or 
    horizontally along (to the right) one square. How many sequences of moves can the robot make to get to the top right square?
\end{exercise}

\newpage

\section{Binomial Coefficients}

\setcounter{statementcount}{1}

We spend this section proving a myriad of identities concerning binomial coefficients, often providing both an algebraic proof and 
a combinatorial proof.

\begin{lemma}[]{addition formula}
    Let $n \in \mathbb{N}$ and $1 \leq k \leq n$. The identity \[\binom{n}{k} = \binom{n-1}{k-1} + \binom{n-1}{k}\] holds.
\end{lemma}

\begin{proof}[Proof 1]
    A simple algebraic verification works, I leave this to the reader.
\end{proof}

We can also prove this identity with the technique of ``double counting''. Intuitively, this is just saying that if we can count things in two 
ways they must be equal. For example, if we had an $n \times n$ grid full of numbers, we could count the sum of all the numbers by summing along 
the rows or the columns. For certain selections of numbers, this can yield interesting equalities.  

\begin{proof}[Proof 2]
    We count the $k$-subsets of $[n]$ in two ways. First, all at once, getting $\binom{n}{k}$ such sets. Secondly, by considering seperately the 
    cases when $1$ is in our subset and isn't. This gives $\binom{n-1}{k-1}$ choices with $1$ and $\binom{n-1}{k}$ choices without, giving us the 
    desired equality. 
\end{proof}


\begin{theorem}[]{Binomial theorem}
    For all integers $n \geq 0$ and $x, y \in \mathbb{C}$, one has
    \[(x + y)^n = \sum_{k=0}^n \binom{n}{k}x^ky^{n-k}\]
\end{theorem}

\begin{proof}[Proof 1]
    We first use a straight forward induction argument. Observe, assuming the hypothesis for some $n \geq 0$ (and noting it trivially holds for $n=0$),
    \begin{align*}
        (x+y)^{n+1} &= (x+y)(x+y)^n \\
        &= (x+y)\sum_{k=0}^n \binom{n}{k}x^k y^{n-k} \\
        &= \sum_{k=0}^n \binom{n}{k}x^{k+1}y^{n-k} + \sum_{k=0}^n \binom{n}{k}x^k y^{n-k+1} \\
        &= \sum_{k=0}^{n+1} \left(\binom{n}{k-1} + \binom{n}{k}\right)x^ky^{n - k + 1} \qquad \qquad (\dagger)\\
        &= \sum_{k=0}^{n+1}\binom{n+1}{k}x^ky^{(n+1)-k}
    \end{align*}
    where $(\dagger)$ follows from an index shift ($k \to k-1$) and defining $\binom{n}{k} := 0$ for $k > n$ or $k < 0$.
\end{proof}

As before, we can double count ourselves a proof of this theorem. 

\begin{proof}[Proof 2]
    Consider the coefficient of $x^ky^{n-k}$ of \[(x+y)^n = \overbrace{(x+y)\cdots(x+y)}^{n \text{ times}}\] When computing this expansion, 
    we are multiplying exactly one of $x$ or $y$ from each $(x+y)$ term on the RHS. Our entire expansion will be the sum of all such choices. 
    Thus, to compute the coefficient of $x^ky^{n-k}$ in $(x+y)^n$, it suffices to count the number of ways to choose exactly $k$ $x$'s out the 
    $n$ possible choices. This is clearly $\binom{n}{k}$ and hence the result follows. 
\end{proof}

As an immediate corollary we have the following.

\begin{corollary}[]{sum and alternating sum of binomial coefficients}
    \[\sum_{k=0}^n \binom{n}{k} = 2^n \qquad \text{and} \qquad \sum_{k=0}^n (-1)^k \binom{n}{k} = 0\]
\end{corollary}

\begin{proof}[Proof 1]
    Take $x=y=1$ and $x=1, y=-1$ in the Binomial Theorem. 
\end{proof}

We may also take a more combinatorial approach to proving these identities. 

\begin{proof}[Proof 2]
    There are $2^n$ subsets of $[n]$ by counting directly (see exercise 1.2) and by counting each of the subsets size $k$ in groups we get 
    $\sum_{k=0}^n \binom{n}{k}$, giving the first equivalence. For the second, note there are an equal number of odd and even subsets of $[n]$ 
    (also exercise 1.2) so we have 
    \[\sum_{k=0}^n (-1)^k \binom{n}{k} = \sum_{k=0}^n\sum_{\substack{k=0, \\ 2 \mid k}}^n \binom{n}{k} - \sum_{\substack{k=0, \\ 2 \mid k}}^{n} \binom{n}{k} = 0\]
    giving the latter equivalence.
\end{proof}

Let us recall the following lemma from (elementary) complex analysis, concerning sums of powers of roots of unity.

\begin{lemma}[]{sums of powers of roots of unity}
    Let $1, \xi_1, \dots, \xi_{n-1}$ be the $n$-roots of unity, that is the solutions to $z^n = 1, z \in \mathbb{C}$. Then, one has 
    \[1 + \sum_{j=1}^{n-1} \xi_j^\ell = \begin{cases}n : \ell \equiv 0 \mod n \\ 0 : \ell \not\equiv 0 \mod n\end{cases}\]
\end{lemma}

\begin{proof}
    Write $\xi_j = \exp(2\pi i j/n)$. If $\ell = mn, m \in \mathbb{Z}$, then $\xi_j^\ell = \exp(2\pi i m) = 1$ and hence
    \[1 + \sum_{j=1}^{n-1}\xi_j^\ell = n\] 
    Conversely, if $\ell = mn + r, m \in \mathbb{Z}, 0 < r < n$ then $\xi_j^\ell = \exp(2\pi i (mn+r)j/n) = \exp(2\pi i jr/n)$ and 
    \[1 + \sum_{j=1}^{n-1} \xi_j^\ell = \sum_{j=0}^{n-1}\left(\exp(2\pi i r/n)\right)^j = \frac{1 - \exp(2\pi i r)}{1 - \exp(2\pi i r/n)} = 0\]
    where we used the standard result for the sum of a geometric series in the second equality. 
\end{proof}

With this we can solve the following problem, a generalisation of corollary 1.3.

\begin{problem}[]{sums of binomial multiples}
    Compute the following quantity. 
    \[\sum_{k=0}^n \binom{mn}{mk}\]
\end{problem}

\begin{proof}[Solution]
    Let $\xi_0, \dots, \xi_{m-1}$ be the $m^\text{th}$ roots of unity, with $\xi_0 = 1$. Then, by our previous lemma and the Binomial theorem,
    we can deduce 
    \[\sum_{k=0}^{n}\binom{mn}{mk} = \sum_{k=0}^{mk}\binom{nm}{k}\left(\frac{1}{m}\sum_{\ell = 0}^{m-1}\xi_\ell^k\right) 
    = \frac{1}{m}\sum_{\ell = 0}^{m-1} \sum_{k=0}^{mn} \binom{mn}{k}\xi_i^k = \frac{1}{m}\sum_{\ell = 0}^{m-1} (1+\xi_\ell)^{mn}\]
    We now use the fact that roots of unity appear in conjugate pairs to further simplify the RHS. Let $z = x + iy \in \mathbb{C}$
    and $n \in \mathbb{N}$. Then \begin{align*}
        z^n + \bar{z}^n &= (x + iy)^n + (x - iy)^n \\ 
        &= \sum_{i = 0}^n x^i (iy)^{n-i} + \sum_{i = 0}^n (-1)^i x^i (iy)^{n-i} \\ 
        &= \sum_{i = 0}^{\lfloor n/2 \rfloor} (-1)^i x^{2i} y^{n - 2i} \\ 
        &= y^n \sum_{i = 0}^{\lfloor n/2 \rfloor} \left(\frac{x^2}{y^2}\right)^i = y^n 
        \frac{\left(\frac{x^2}{y^2}\right)^{\lfloor n/2 \rfloor + 1} - 1}{\left(\frac{x^2}{y^2}\right) - 1}
    \end{align*}
    Now by Euler's formula, we have $e^{i\theta} = \cos \theta + i \sin \theta$, so we may substitute in 
    $x = 1 + \cos\theta$ and $y = \sin \theta$ for $\theta = 2k\pi/n$, $k < n/2$ (so that we are in the top half of 
    the complex plane). For small $m$ (try $m=3$) we will have $\theta$ being so that the exact values of $x$ and $y$ are well 
    known, and hence we are left with something that simplifies nicely. In general though we are left with something 
    rather ugly :(
\end{proof}

\begin{proposition}[]{Vandermonde's Identity}
    Fix $n, a, b, \in \mathbb{N}$. Then 
    \[\sum_{k=0}^n \binom{a}{k}\binom{b}{n-k} = \binom{a + b}{n}\]
\end{proposition}

We observe two proofs, one algebraic and one via double counting. \\ 

The motivation for this first proof comes from the fact we are taking a sum of the form $\sum_{k=0}^n a_kb_{n-k}$, 
which should remind us of the multiplication of polynomials. 

\begin{proof}[Proof 1]
    We work by comparing the coefficients of $(1+x)^{a+b} = (1+x)^a(1+x)^b$. By the Binomial theorem, one has
    \[(1+x)^a (1+x)^b = \left(\sum_{k_1=0}^a \binom{a}{k_1}x^{k_1}\right)\left(\sum_{k_2=0}^b \binom{b}{n - k_2}x^{k_2}\right)
    = \sum_{n=0}^{a+b}\left(\sum_{k=0}^n \binom{a}{k}\binom{b}{n-k}\right)\]
    where the last equality comes from computing the polnomial multiplication. Now, applying the Binomial theorem again, see 
    \[(1+x)^{a+b} = \sum_{n=0}^{a+b}\binom{a+b}{n}x^k\]
    so by the comparing coefficients of these two polynomials we deduce the result.
\end{proof}

A nice way to think about this double counting proof is to count the ways of forming a commitee of $n$ people from 
$a$ boys and $b$ girls, choosing $k$ boys at a time.

\begin{proof}[Proof 2]
    We count the number of ways to choose subsets size $n$ from $[a+b]$ in two ways. The first, directly, gives 
    $\binom{a+b}{n}$ choices. The second, is by counting the number of ways to choose such a subset with exactly 
    $0 \leq k \leq n$ elements in $[a]$. We have $\binom{a}{k}$ ways of choosing the $k$ elements in $[a]$, and 
    $\binom{b}{n-k}$ ways of choosing the remaining $n-k$ elements from $[a+b] \setminus [a]$. Thus, summing over 
    $0 \leq k \leq n$ we obtain the desired equality.
\end{proof}

\begin{proposition}[]{hockeystick lemma}
    For $n, r \in \mathbb{N}$ with $n \geq r$, one has \[\sum_{k=r}^n \binom{k}{r} = \binom{n+1}{r+1}\]
\end{proposition}

As usual, there are plenty of ways of proving this. We'll do one algebraic and one combinatorial. 

\begin{proof}[Proof 1]
    We use the Binomial addition formula to transform our sum into a telescoping series. Observe, 
    \begin{align*}
        \sum_{k = r}^n \binom{k}{r} = \sum_{k=r}^n \left[\binom{k+1}{r+1} - \binom{k}{r+1}\right] 
        = \sum_{k = r+1}^{n+1}\binom{k}{r+1} - \sum_{k = r}\binom{k}{r+1} = \binom{n+1}{r+1} - \binom{r}{r+1} 
        = \binom{n+1}{r+1}
    \end{align*}
    where the first equality is a consequence of Lemma 1.2.1. 
\end{proof}

This next proof uses the so called ``stars and bars'' method that we encountered last chapter to count the 
arrangements of $n$ indistinguishable balls into $k$ distinguishable boxes. 

\begin{proof}[Proof 2]
    We work by double counting the arrangements of $n$ indistinguishable red balls into $k$ 
    indistinguishable blue balls into $k$ distinguishable boxes. By ``stars and bars'' we have 
    \[\binom{n+k-1}{k-1}\]
    such arrangements. Now instead, label the boxes $1, \dots, k$ and count the arrangements with $0 \leq i \leq k$ 
    blue balls in box $1$. There are 
    \[\binom{(n+k-2) - i}{k-2}\]
    such arrangements for each $i$ and hence 
    \[\binom{n+k-1}{k-1} = \sum_{i = 0}^k \binom{(n+k-2) - i}{k-2}\]
    Taking $n^\prime = n + k - 2$ and $r = k-2$ we recover 
    \[\binom{n^\prime+1}{r + 1} = \sum_{i=r}^{n^\prime}\binom{i}{r}\]
    by reversing the order of summation, which as $k$ and $n$ were arbitrary concludes the proof.
\end{proof}

\subsection*{Exercises}

\begin{exercise}[]{}
    How many words length $n$ can be formed from an alphabet of $\ell$ letters $\mathcal{A} = \{A_1, \dots, A_\ell\}$ such that the first letter 
    $A_1$ occurs an even number of times?
\end{exercise}

\begin{exercise}[]{}
    \begin{enumerate}[(i)]
        \item Using induction, find another proof for the hockeystick lemma.
        \item By assigning $n-k+1$ labels to the elements of $[n+1]$ or otherwise, find another double counting 
        proof of the hockeystick lemma.
    \end{enumerate} 
\end{exercise}

\begin{exercise}[]{} 
    Compute \[\sum_{k=0}^n k\binom{n}{k} \quad \text{and} \quad \sum_{k=0}^n k^2\binom{n}{k}\] 
    algebraically and combinatorially. \\ 
    ($\star$) Can you find probabilistic proofs of these facts? 
\end{exercise}

\section{The Pigeonhole Principle}

\setcounter{statementcount}{1}

If I have $5$ pigeons and $4$ containers, each only able to fit one pigeon, can I fit all the pigeons into my containers? Of course not! It turns 
out this common-sense principle allows us to discover many, many, combinatorial facts$\dots$

\begin{theorem}[]{pigeonhole principle}
    Given a set $X$ of size $n$, any partition of $X$ into $m < n$ subsets $X_1, \dots, X_m$ must have at least one $X_i$ with $|X_i| > 1$. 
\end{theorem}

We prove this with the following simple contradiction. 

\begin{proof}
    Suppose each $|X_i| \leq 1$, then $|X| = |\cup_{i=1}^m X_i| = \sum_{i=1}^m |X_i| \leq m < n$. Absurd!
\end{proof}

We can do slightly better than this. What if my containers can fit $2$ pigeons and I only have $2$ this time. Then I still can't fit my pigeons 
into my containers. Formally, 

\begin{theorem}[]{full pigeonhole principle}
    Given a set $X$ of size $n = km+1$, any partition of $X$ into $m$ subsets $X_1, \dots, X_m$ must have at least one $X_i$ with $|X_i| > k$.
\end{theorem}

\begin{proof}
    Suppose each $|X_i| \leq k$, then $|X| = |\cup_{i=1}^m X_i| = \sum_{i=1}^m |X_i| \leq km < n$. Absurd!
\end{proof}

We also have a infinite pigeonhole principle. If I can only fit a finite number of pigeons into each container then, assuming I only have 
finitely many containers, I cannot fit an infinite number of pigeons into my containers.

\begin{theorem}[]{infinite pigeonhole principle}
    Given a set $X$ of infinite cardinality, any partition of $X$ into finitely many sets $X_1, \dots, X_m$ must have some $X_i$ also of infinite 
    cardinality. 
\end{theorem}

\begin{proof}
    Suppose each $|X_i| = n_i < \infty$. Then $|X| = |\cup_{i}^m X_i| = \sum_{i=1}^m |X_i| 
    = \prod_{i=1}^m n_i < \infty$. Absurd!
\end{proof}

\begin{problem}[]{monotone subsequences}
    How large must a sequence of distinct real numbers be to guarantee the existence of a monotone subsequence 
    size $n + 1$, $n \in \mathbb{N}$? 
\end{problem}

\begin{proof}[Solution]

\end{proof}

\subsubsection*{Exercises}

\newpage

\section{The Principle of Inclusion-Exclusion}

\setcounter{statementcount}{1}

It is often much easier to count how many objects have properties $A$ AND $B$ than $A$ OR $B$. Thankfully we have the following principle to 
relate the two.

\begin{proposition}[]{2 variable inclusion exclusion principle}
    Let $A$ and $B$ be finite sets. Then, 
    \[|A \cup B| = |A| + |B| - |A \cap B|\]    
\end{proposition}

For an intuitive argument, just draw a venn diagram. To find the count $A \cup B$ we can count the entire circles
of $A$ and $B$ individually, but when doing this we count the $A \cap B$ section twice so we must subtract off one 
lot of it. \\

We prove this by showing for each $x \in A \cup B$, $x$ is counted exactly once in the RHS. 

\begin{proof}
    Write $|A \cup B| = \sum_{x \in A \cup B}1$.  WLOG take $x \in A$, then either $x \in B$ or $x \not\in B$. In the former 
    case, $x \in A \cap B$ gives a count of exactly $1$ on the RHS for $x$ and in the latter case $x \not\in A \cap B$ 
    affirms the same result. Hence $|A| + |B| - |A \cap B| = \sum_{x \in A \cup B}1 = |A \cup B|$. 
\end{proof}

It turns out this property can be generalised for $n$ finite sets.

\begin{theorem}[]{inclusion exclusion principle}
    Let $n \in \mathbb{N}$ and $A_1, \dots, A_n$ be finite sets. Then, 
    \[\left|\bigcup_{i=1}^n A_i\right| = \sum_{\ell = 1}^n (-1)^{\ell-1}\sum_{I \subseteq [n], |I| = \ell} \left|\bigcap_{i \in I}A_i\right|\]
\end{theorem}

We have two proofs for this. A bashy induction will work, but there is also a clever little trick we can do with indicator functions. 

\begin{proof}[Proof 1]
    We work via induction$\dots$
\end{proof}

\begin{proof}[Proof 2]
    Suppose that all of our subsets lie in a space $\Omega$ and denote $A^c := \Omega \setminus A$ for $A \subseteq \Omega$. Then, 
    \[{\bf 1}\left[\bigcup_{i=1}^n A_i\right] = {\bf 1}\left[\left(\bigcap_{i=1}^n A_i^c\right)^c\right]
    = 1 - {\bf 1}\left[\bigcap_{i=1}^n A_i^c\right] = 1 - \prod_{i=1}^n (1 - {\bf 1}[A_i])\]
    Where ${\bf 1}[A] = 1$ if $x \in A$ and $0$ elsewhere (we call this the indicator function). Now, expanding the 
    product we see 
    \[{\bf 1}\left[\bigcup_{i=1}^n A_i\right] 
    = \sum_{\ell = 1}^n \sum_{I \subseteq [n], \\ |I| = \ell} \, \prod_{i \in I} (- {\bf 1}[A_i])\] 
    which, summing both sides over $x \in \Omega$, gives the result.
\end{proof}

We may count, for $n$ divisible by a prime $p$, the number of integers less than or equal to $n$ divisible by $p$ as $n/p$. 
Recall  that for primes $p, q$ and $n \in \mathbb{N}$, one has $p, q \mid n \Leftrightarrow pq \mid n$. Thus, it is easy to count how 
many naturals below $n$ are divisible by $p$ and $q$ (just biject), and by the inclusion exclusion principle it must also be easy to 
count how many naturals below $n$ are divisible by $p$ or $q$. \\

Recall that {\it Euler's totient function} $\phi: \mathbb{N} \to \mathbb{N}$ has $\phi(n) = |\{1 \leq i \leq n : \text{gcd}(i, n) = 1\}|$. 
That is, $\phi$ counts the number of integers at most $n$ which are coprime to $n$. We will count this quantity directly, working with 
the reasoning outlined above.

\begin{proposition}[]{explicit form of Euler's $\phi$-function}
    Let $n \in \mathbb{N}$. Then, letting $\phi$ be Euler's totient function, 
    \[\phi(n) = n \! \! \! \prod_{\substack{p \mid n, \\ p \text{ prime}}} \! \! \! \left(1 - \frac{1}{p}\right)\]
\end{proposition}

\begin{proof}
    Factorise $n = p_1^{\ell_1} \cdots p_k^{\ell_k}$ into primes. Then we have $1 \leq x \leq n$ coprime to $n$ 
    if and only if $p_1, \dots, p_k \nmid x$. Define $A_p := \{1 \leq x \leq n : p \mid x\} \subseteq [n]$. 
    By the inclusion exclusion principle we may compute, where compliments are taken with respect to $[n]$, 
    \begin{align*}
        \phi(n) &= \left|\bigcap_{i=1}^kA_{p_i}^c\right| = n - \left|\bigcup_{i=1}^k A_{p_i}\right| \tag{1}\\
        &= n - \sum_{m = 1}^k (-1)^{m-1} \sum_{I \subset [k], |I| = m}\left|\bigcap_{i \in I}A_{p_i}\right| \tag{2}\\
        &= n - \sum_{m = 1}^k (-1)^{m-1} \sum_{I \subset [k], |I| = m}\frac{n}{\prod_{i \in I}p_i} \tag{3}\\
        &= n\sum_{m = 0}^k (-1)^{m} \sum_{I \subset [k], |I| = m}\frac{1}{\prod_{i \in I}p_i} \tag{4}
        = n\prod_{p \mid n}\left(1 - \frac{1}{p}\right)
    \end{align*}
    Where in $(1)$ we used DeMorgan's laws, $(2)$ the inclusion exclusion principle, in $(3)$ the fact that 
    $|\cap_{p \in P}A_p| = n/\prod_{p \in P}$ when $p \in P$ are primes dividing $n$ (easily seen with 
    $pq \mid n \Leftrightarrow p,q \mid n$) and in (4) the fact $\prod_{i \in I}(1-x_i) = \sum_{J \subseteq I}(-1)^{|J|}\prod_{j \in J}x_j$, 
    where the empty product is defined as $1$. 
\end{proof}

We can also use the inclusion exclusion principles to count the number of permutations of $[n]$ with no fixed point, 
the so called {\it dearrangements} of $[n]$. A fixed point of a permutation $\pi:[n] \to [n]$ is simply an 
$x \in [n]$ with $\pi(x) = x$. 

\begin{proposition}[]{dearrangements}
    The number of permutations of $[n]$ with no fixed point is \[n!\sum_{k = 0}^n \frac{(-1)^k}{k!}\]
\end{proposition}

\begin{proof}
    This is another direct application of the inclusion exclusion principle. Let $A_i$ be the set of permutations 
    on $[n]$ with $i$ a fixed point. Then we compute
    \[
        \left|\bigcap_{i=1}^n A_i^c\right| = n! - \left|\bigcup_{i = 1}^n A_i\right| = n! - \sum_{k=1}^n
        \sum_{I \subset [n], |I| = k}\left|\bigcap_{i \in I}A_i\right| = n! - \sum_{k=1}^n (-1)^{k-1}\binom{n}{k}(n-k)! 
        = n! \sum_{k=0}^n \frac{(-1)^k}{k!}
    \]
    using the fact there are $\binom{n}{k}$ subsets $I \subseteq [n]$ of size $k$ and $(n-k)!$ permutations 
    with $k$ elements fixed. 
\end{proof}

If we take the limit $n \to \infty$, we see that the probability of a (uniformly) random permutation of $[n]$ having
no fixed points tends to $1/e$. 

\newpage

\section{Recursion I}

\setcounter{statementcount}{1}

\subsection*{Exercises}

\newpage

\section{Recursion II}

\setcounter{statementcount}{1}

\newpage 

\section{Miscellaneous Topics}

\subsubsection*{Exercises}

\newpage

\section{Additional Problems}

Here I list the exercise solutions in a random order. I will store which exercises have a solution in the README of this repository. 

\begin{proof}[Solution 1.9]
    We observe that the table has rotational symmetry as the table is round, so that each $n!$ ways we could arrange the people in a line 
    could sit in $n$ possible ways giving $(n-1)!$ total seatings. 
\end{proof}

\begin{proof}[Solution 1.10]
    For the first part I give three solutions.
    \begin{enumerate}
        \item We write the set as a binary string. 
        Map $\mu_A(i) = 1$  if $i \in A$ and $0$ elsewhere in $[n]$ for each subset $A \subseteq [n]$. Then there as each $\mu_A$ 
        can either have $i = 0$ or $i = 1$ for $i \in [n]$ we have $2^n$ possible choices for the $\mu_A$ which bijects with the subsets 
        of $[n]$. $\qquad \square$
        \item We induct. 
        Clearly there are $2$ subsets of $[1]$, $\emptyset$ and $\{1\}$. Suppose there are $2^{n-1}$ subsets of $[n-1]$. Then for each subset
        of $[n]$, either $n \in [n]$ or $n \notin [n]$ so we have $2 \times 2^{n-1} =2^n$ subsets of $[n]$ and the induction is complete. $\qquad \square$
        \item For each $0 \leq k \leq n$ there are $\binom{n}{k}$ subsets of size k so by the binomial theorem there are 
        $\sum_{k = 0}^n \binom{n}{k} = 2^n$ such subjects. $\qquad \square$ 
    \end{enumerate}
    For the second, two. 
    \begin{enumerate}
        \item We instead biject from the odd subsets to the even. Consider the map $\mu(A)$ from the subsets of $[n-1]$ to the even sized 
        subsets of $[n]$ that has $A \mapsto A \cup \{n\}$ iff $|A|$ odd and to itself otherwise. Then $\mu(A)$ has a clear inverse 
        mapping $A \mapsto A \setminus \{n\}$ iff $|A|$ odd and to itself if $|A|$ even. Hence as $\mu$ bijects we have $2^{n-1}$ such 
        subsets from the first part. $\qquad \square$
        \item We have 
        \[\sum_{\substack{A \subseteq [n] \\ 2 \;  \mid \; |A|}}\binom{n}{|A|} - \sum_{\substack{A \subseteq [n] \\ 2 \; \nmid \; |A|}}\binom{n}{|A|}
        = \sum_{A \subseteq [n]} \binom{n}{|A|}(-1)^{|A|} = 0\] 
        from the binomial theorem and hence an equal number of odd and even sized subsets, giving $2^n / 2 = 2^{n-1}$ odd/even subsets. $\qquad \square$
    \end{enumerate}
    I'm sure more proofs of these facts exist. Try to come up with some!
\end{proof}

\begin{proof}[Solution 1.11]
    Select the pairs sequentially, giving $(n \times (n-1)) \times (n-1) \times (n-2) \times \cdots \times 2 \times 1$ 
    possible ways of selecting the people, and $(n!)^2$  
\end{proof}

\bibliographystyle{plain} 
\bibliography{refs} 

\end{document}