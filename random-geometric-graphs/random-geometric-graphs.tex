\documentclass{article} 

\usepackage[most]{tcolorbox}
\usepackage{xcolor}
\usepackage{enumerate}
\usepackage[a4paper, margin=1in]{geometry} 
\usepackage{titling}
\usepackage{amssymb}
\usepackage{lipsum}
\usepackage{mathtools}
\usepackage{amsthm} % For proof environments only, labelling of thms done with custom counter
\usepackage{amsmath}
\usepackage{amssymb}
\usepackage{textcomp} % TM logo, shits n giggles
\usepackage{hyperref}

%%% Misc commands %%%
\newcommand\iidsim{\stackrel{\mathclap{\tiny\mbox{i.i.d}}}{\sim}}
\newcommand\indist{\stackrel{\mathclap{\tiny\mbox{$\mathcal{D}$}}}{\longrightarrow}}
\newcommand\nidist{\stackrel{\mathclap{\tiny\mbox{$\mathcal{D}$}}}{\longleftarrow}}
\newcommand{\indep}{\perp\!\!\!\!\perp} 

\setlength{\parindent}{0pt} % Remove indentation upon new paragraph. 

%%% Title page information %%%
\title{Random Geometric Graphs}
\author{Jacob Green}
\date{\today}
\newcommand{\subtitle}{}
\newcommand{\institution}{Department of Mathematical Sciences, The University of Bath}
\newcommand{\keywords}{Random Geometric Graphs}

\newcounter{statementcount}
\counterwithin{statementcount}{subsection}
\renewcommand{\thestatementcount}{\thesection.\arabic{statementcount}}
\newcounter{claimcount} % Used for proof with lots of claims

\newtcolorbox{definition}[2][auto counter, number within=section]{
  colback=black!5!white, 
  colframe=black!50!white, 
  fonttitle=\bfseries, 
  coltitle=black,
  title=Definition \thestatementcount $\:$ (#2), 
  before upper = \refstepcounter{statementcount},
  #1, % Optional params
}

\newtcolorbox{lemma}[2][auto counter, number within=section]{
  colback=blue!5!white, 
  colframe=blue!50!white, 
  fonttitle=\bfseries, 
  coltitle=black, 
  title=Lemma \thestatementcount $\:$ (#2),
  before upper = \refstepcounter{statementcount} 
  #1, % Optional params
}

\newtcolorbox{example}[2][auto counter, number within=section]{
  colback=green!5!white, 
  colframe=green!50!white, 
  fonttitle=\bfseries, 
  coltitle=black, 
  title=Example \thestatementcount $\:$ (#2), 
  before upper = \refstepcounter{statementcount}
  #1, % Optional params
}

\newtcolorbox{problem}[2][auto counter, number within=section]{
  colback=purple!5!white, 
  colframe=purple!50!white, 
  fonttitle=\bfseries, 
  coltitle=black, 
  title=Problem \thestatementcount $\:$ (#2), 
  before upper = \refstepcounter{statementcount}
  #1, % Optional params
}

\newtcolorbox{theorem}[2][auto counter, number within=section]{
  colback=red!5!white,
  colframe=red!50!white, 
  fonttitle=\bfseries, 
  coltitle=black, 
  title=Theorem \thestatementcount $\:$ (#2), 
  before upper = \refstepcounter{statementcount},
  #1, % Optional params
}

\newtcolorbox{proposition}[2][auto counter, number within=section]{
  colback=purple!5!white,
  colframe=purple!50!white, 
  fonttitle=\bfseries, 
  coltitle=black, 
  title=Proposition \thestatementcount $\:$ (#2), 
  before upper = \refstepcounter{statementcount},
  #1, % Optional params
}

\newtcolorbox{corollary}[2][auto counter, number within=section]{
  colback=yellow!5!white,
  colframe=yellow!50!white, 
  fonttitle=\bfseries, 
  coltitle=black, 
  title=Corollary \thestatementcount $\:$ (#2),
  before upper = \refstepcounter{statementcount}, 
  #1, % Optional params
}

\newtcolorbox{remark}[2][auto counter, number within=section]{
  colback=black!5!white,
  colframe=black!50!white, 
  fonttitle=\bfseries, 
  coltitle=black, 
  title=Remark \thestatementcount $\:$ (#2),
  before upper = \refstepcounter{statementcount}, 
  #1, % Optional params
}

\newtcolorbox{exercise}[2][auto counter, number within=section]{
  colback=black!5!white,
  colframe=black!50!white, 
  fonttitle=\bfseries, 
  coltitle=black, 
  title=Exercise \thestatementcount $\:$ #2,
  before upper = \refstepcounter{statementcount}, 
  #1, % Optional params
}

\newtcolorbox[use counter=claimcount]{claim}[1][]{%
    colback=green!10, 
    colframe=green!50!black, 
    coltitle=black, 
    fonttitle=\bfseries, 
    boxrule=0.5mm, 
    colbacktitle=green!30,
    enhanced, % Allows additional options
    boxed title style={sharp corners}, 
    attach title to upper={},
    titlerule=0mm, 
    title=Claim: $\;$,
    before=\par\smallskip\noindent, 
    #1
}

\begin{document}

\begin{titlepage}
    \centering
    % Adding an image (optional)
    
    % Title
    {\Huge \bfseries \thetitle \par}
    \vspace{0.5cm}
    
    % Subtitle (if any)
    {\Large \subtitle \par}
    \vspace{1cm}
    
    % Author and institution
    {\large \theauthor \par}
    {\institution \par}
    \vspace{1cm}
    
    % Date
    {\large \thedate \par}
    \vspace{1.5cm}
    
    % Abstract section
    \begin{abstract}
        \lipsum[10]
    \end{abstract}
    \vspace{1cm}
    
    % Keywords section
    \textbf{Keywords:} \keywords
    \vfill % Pushes the following content to the bottom
    
    % Footer or further information
    \textit{}
\end{titlepage}

\newpage 

\setcounter{page}{1} % Start page numbering from 1 after title page

\section*{Preface}

\tableofcontents

\newpage

\section{Introduction}

\setcounter{statementcount}{1}

Consider a (finite) collection of points in space, $X \subset \mathbb{R}^d$. We wish to model the situation 
where points $x, y \in X$ that are ``close'' are connected. In the graph theoretic sense this would be done by 
connecting $x$ and $y$ with an edge. Given we are in Euclidean space (but noting this logic can be extendend to 
arbitrary metric spaces) we can define ``closeness'' with the usual metric on $\mathbb{R}^d$, the Euclidean norm 
$\lVert \cdot \rVert$. Hence we connect $x \sim y$ iff $\lVert x - y \rVert \leq r$ for some $r > 0$. The resulting 
graph, the so called {\it geometric graph}, is denoted $G(X, r)$. \\ 

How do we make this random? In the classical Erd\H{o}s-Renyi random graph our randomness comes from the 
connection criterion on vertices. Here our notion of connectivity is deterministic, so we instead allow the 
points (vertices) $X \subset \mathbb{R}^d$ to be random. Some classic (and the examples we'll study in this 
handout) examples are $X = (X_1, \dots, X_n) \sim \text{Unif}[0,1]^d$ and $X \sim \text{PPP}_{[0,1]^d}(\lambda)$, 
that is $X$ follows a Poisson point process intensity $\lambda$ in the unit hypercube $[0,1]^d$. \\

A classic use case for this model would be the spread of infection in a plant population. If we know roughly 
the density of plants in a given habitat and how close two plants must be for a given disease to spread between 
them, we may model the spread of this disease with a random geometric graph (with Poisson vertices). Then we may 
answer questions like ``What is the maximum number of plants that could be affected by releasing this disease into 
the population?'' by looking at its connected components, and other similar questions by looking at 
graph theoretic properties of the random geometric graph. \\ 

In this handout we are concerned with sequences of random geometric graphs on the hypercube $[0,1]^d$. We will 
work to prove several asymptotic properties of these graphs in the limiting density ($n \to \infty$ or $\lambda 
\to \infty$ in our previous examples). Let us define our sequences of interest. 

\begin{definition}[]{fixed/poisson vertex random geometric graphs}
    Let $n \in \mathbb{N}$ and let $\xi_1, \xi_2, \dots$ be independent points in $\mathbb{R}^d$. Let $N_n \sim \text{Po}(n)$ 
    independently of $\xi_1, \xi_2, \dots$ and consider the two point sets 
    \[\mathcal{X}_n = \{\xi_1, \dots, \xi_n\} \qquad \text{and} \qquad \mathcal{P}_n = \{\xi_1, \dots, \xi_{N_n}\}\]
    Let $(r_n)_{n \geq 1}$ be a sequence of positive reals. Then we define, respectively, for $n \geq 1$ the
    {\it fixed vertex random geometric graphs} and {\it Poisson vertex random geometric graphs} as 
    \[G(\mathcal{X}_n, r_n) \qquad \text{and} \qquad G(\mathcal{P}_n, r_n)\]
\end{definition}

We will study properties as $n \to \infty$ in the various cases on the size of $r_n$ as $n \to \infty$. The three 
types of limit in particular we care about are the so called {\it sparse limit} where $nr_d^n \to 0$, the 
{\it thermodynamic limit} $nr_n^d = \Theta(1)$ and the {\it dense limit} $nr_n^d \to \infty$. \\ 

It is not immediately obvious that our Poisson vertex random geometric graph $G(\mathcal{P}_n, r_n)$ corresponds 
to a Poisson point process in $[0,1]^d$, as suggested earlier. Let us prove this fact. 

\begin{proposition}[]{fixed vertex RGG is a Poisson point process}
    The set $\mathcal{P}_n$ defines a Poisson point process in $[0,1]^d$. That is, for Borel $A, A_1, \dots, A_k \in [0,1]^d$ 
    \begin{enumerate}[(I)]
        \item $\mathcal{P}_n(A) \sim \text{Po}(n\lambda(A))$
        \item $\mathcal{P}_n(A_1), \dots, \mathcal{P}_n(A_k)$ are independent random variables for $A_1, \dots, A_k$ disjoint
    \end{enumerate}
    where $\mathcal{P}_n(X)$ denotes the number of points in $\mathcal{P}_n \cap X$ and $\lambda(\cdot)$ is the Lebesgue 
    measure in $\mathbb{R}^d$. 
\end{proposition}

\begin{proof}
    Each $\xi_i$ is in $A \subset [0,1]^d$ with probability $\lambda(A)$. Thus, conditioned on $N_n = m$, $\mathcal{P}_n(A)$ 
    is a $\text{Binom}(m, \lambda(A))$. Further, considering disjoint $A_1, \dots, A_k \subset [0,1]^d$ and 
    conditioning on $N_n = m$, we see that the random vector $(\mathcal{P}_n(A_1), \dots, \mathcal{P}_n(A_k))$ is 
    $\text{Multinom}(k, (\lambda(A_1), \dots, \lambda(A_k)))$. Suppose now that $A_1, \dots, A_k$ partition $[0,1]^d$, 
    then, letting $m = \sum_{i=1}^k j_k$,
    \begin{align*}
        \mathbb{P}[\mathcal{P}_n(A_1) = j_1, \dots, \mathcal{P}_n(A_k) = j_k] &= 
        \underbrace{\left(\frac{e^{-n}n^m}{m!}\right)}_{\text{Po}(m)} \times 
        \underbrace{\binom{m}{j_1, \dots, j_k}\prod_{i=1}^k \lambda(A_i)^{j_i}}_{\text{Multinom}(k, (\lambda(A_1), \dots, \lambda(A_k)))} \\ 
        &= \prod_{i=1}^k \underbrace{\frac{e^{-n\lambda(A_i)}(n\lambda(A_i))^{j_i}}{j_i!}}_{\text{Po}(n\lambda(A_i))}
    \end{align*}
    are independent $\text{Po}(n\lambda(A_i))$, which demonstrates the result.
\end{proof}

Before moving on to edge counts, let us prove one elementary property of random geometric graphs to get a feel 
for these objects. 

\begin{proposition}[]{expected vertex degrees}
    Let $D_{i, n}$ denote the degree of vertex $\xi_i$ in $G(\mathcal{X}_n, r_n)$ and $D_{i, n}^\prime$ the degree 
    of vertex $i$ in $G(\mathcal{X}_n, r_n)$ suppose $r_n \to 0$. Then 
    \[\mathbb{E}[D_{i, n}], \mathbb{E}[D^\prime_{i, n}] \sim \theta_d n r_n^d\]
    as $n \to \infty$, where $\theta_d$ is the volume of the unit ball in $\mathbb{R}^d$.
\end{proposition}

Note that due to points near the boundary we do not get equality. If we were to condition on the event $B_{\xi_i}(r_n) 
\subset [0,1]^d$ for $n \geq 1$ we would have equality.

\begin{proof}
    The idea is to use the law of total expectation and the dominated convergence theorem. Observe, conditioning on 
    $\xi_i = x \in \mathbb{R}^d$, that $D_{i, n} \sim \text{Binom}(n - 1, \lambda(B_x(r_n) \cap [0,1]^d))$ where 
    $B_x(r_n)$ is the $d$-ball centered at $x$ with radius $r_n$ under the Euclidean norm. Observe,
    \begin{align*}
        r_n^{-d}\mathbb{E}[D_{i,n}] &=  
        \int_{x \in [0,1]^d} r_n^{-d}\mathbb{E}[D_{i, n} | \xi_i = x]\mathrm{d}x \\
        &= (n-1)\int_{x \in [0,1]^d} r_n^{-d} \lambda(B_x(r_n) \cap [0,1]^d)\mathrm{d}x \\
        &\sim n\theta_d
    \end{align*}
    We apply the dominated convergence theorem in the asymptotic equality, noting that \[r_n^{-d} \lambda(B_x(r_n) \cap [0,1]^d) 
    \to \theta_d\] almost everywhere in $[0,1]^d$ and is clearly dominated by $1$. The result follows immediately. \\
\end{proof}

\begin{remark}[]{applying the dominated convergence theorem in 1.3}
    We multiply through by $r_n^{-d}$ or our integrand $\lambda(B_x(r_n) \cap [0,1]^d)$ will tend pointwise to $0$ 
    and we will get $n^{-1}\mathbb{E}[D_{i, n}] \to 0$ which is true (assuming $r_n \to 0$) but uninteresting. 
\end{remark}


\section{Edge Counts}   

Throughout this section we will denote the number of edges of $G(\mathcal{X}_n, r_n)$ by $\mathcal{E}_n$ and the 
number edges of $G(\mathcal{P}_n, r_n)$ by $\mathcal{E}^\prime_n$. Our goal is to find the limiting distribution of 
these two quantities. \\

Before doing so, we'll first investigate the expected number of edges in such a graph. We apply a classical lemma 
from graph theory to find the expected number of edges in $G(\mathcal{X}_n, r_n)$. Then for the expected number of 
edges in $G(\mathcal{P}_n, r_n)$ we'll use properties of the conditional expectation.

\begin{proposition}[]{expected edge counts}
    In the notation outlined above, 
    \[\mathbb{E}[\mathcal{E}_n], \mathbb{E}[\mathcal{E}_n^\prime] \sim n^2 r_d^n \theta_d / 2\]
\end{proposition}

\begin{proof}
    By the handshaking lemma, linearity of expectations and proposition 1.3, one has 
    \[2\mathbb{E}[\mathcal{E}_n] = \sum_{i=1}^n \mathbb{E}[D_{1, n}] \sim n^2 \theta_d r_n^d\]
    and the result follows by dividing through by $2$. Now by the total law of expectations one has 
    \[\mathbb{E}[\mathcal{E}^\prime_n] = \mathbb{E}[\mathbb{E}[\mathcal{E}_n^\prime | N_n = m]] 
    = \]
\end{proof}

\subsection{Edge Distribution}

\subsection{Normal Approximation}

\section{Maximal Degree}

\end{document}